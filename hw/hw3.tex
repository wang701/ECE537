%-----------------------------------------------------------------------------
%	PACKAGES AND DOCUMENT CONFIGURATIONS
%-----------------------------------------------------------------------------

\documentclass{article}

\usepackage{graphicx} % Required for the inclusion of images
\usepackage{natbib} % Required to change bibliography style to APA
\usepackage{amsmath} % Required for some math elements
\usepackage{amssymb}
\usepackage{grffile}
\usepackage[export]{adjustbox}
\usepackage{subcaption}
\usepackage{float}
\usepackage{listings}
\usepackage[margin=1.0in]{geometry}
\usepackage{tikz}
\usepackage{enumitem}
\usepackage{scrextend}
\usepackage{siunitx}
\usepackage{minted}

\usetikzlibrary{shapes.geometric, arrows}
\tikzstyle{startstop} = [rectangle, rounded corners, minimum width=1cm, minimum height=1cm,text centered, draw=black, fill=white!30]
\tikzstyle{process} = [rectangle, minimum width=1cm, minimum height=1cm, text centered, draw=black, fill=white!30]
\tikzstyle{arrow} = [thick,->,>=stealth]

\setlength\parindent{0pt} % Removes all indentation from paragraphs

%-----------------------------------------------------------------------------
%	DOCUMENT INFORMATION
%-----------------------------------------------------------------------------

\title{ECE 547 Fall 2016 Homework 3} % Title

\author{Yang \textsc{Wang}}  % Author name

\date{\today} % Date for the report

\renewcommand{\theenumi}{\alph{enumi}} % use letters for list items

\begin{document}

\maketitle % Insert the title, author and date

%-----------------------------------------------------------------------------
%	Problem 1
%-----------------------------------------------------------------------------

\section*{Analytical Questions}
	\subsection*{Problem 1}

		Propagation Delay: the amount of time needed for traveling to McDonald's and
		back to the office. \\
		Transmission Delay: the amount of time needed for consuming the food. \\
		Processing Delay: the amount of time needed for choosing the meal. \\
		Queueing Delay: the amount of time needed for waiting up in line to order.

	\subsection*{Problem 2}
	%TODO

	\subsection*{Problem 3}

		Let $A = \left\{ \text{one packet arrive at the node successfully} \mid \text{packet has length } n \right\}$,
		\begin{gather*}
			\implies P(A) = (1-p)^{n}
		\end{gather*}

		Let $B = \left\{ \text{packets arrive at the node successfully} \right\}$
		and $C = \left\{ \text{packet has length } n \right\} $,
		\begin{align*}
			\implies P(B) &= \sum_{n = 0}^{\infty} P(A) \cdot P(C) \\
			&= \sum_{n = 0}^{\infty} (1-p)^{n} \frac{\mu^{n}e^{-\mu}}{n!} \\
			&= e^{-\mu} \sum_{n = 0}^{\infty} (1-p)^{n} \frac{\mu^{n}}{n!} \\
			&= e^{-\mu p} \sum_{n = 0}^{\infty} e^{-\mu (1-p)} \frac{((1-p)\mu)^{n}}{n!} \\
			&= e^{-\mu p} \cdot 1 \\
			&= e^{-\mu p}
		\end{align*}

		Therefore, the rate at which succcessful packets arrive the network node is:
		\begin{align*}
			\text{rate} = e^{-\mu p} \cdot \lambda
		\end{align*}

	\section*{Problem 4}

		We are given that $\tau_{1}$ and $\tau_{2}$ are independent, exponential
		distribution with mean $\frac{1}{\lambda_{1}}$ and $\frac{1}{\lambda_{2}}$.
		Therefore, we can write,
		\begin{align*}
			P(\left\{ \tau_{1}  \geqslant t \right\}) &= e^{-\lambda_{1} t} \\
			P(\left\{ \tau_{2}  \geqslant t \right\}) &= e^{-\lambda_{2} t}
		\end{align*}
		We can also write,
		\begin{align*}
			P(\left\{ \text{min}(\tau_{1}, \tau_{2}) \right\}) &= P(\left\{ \tau_{1} \geqslant t, \tau_{2} \geqslant t  \right\}) \\
			&= P(\left\{ \tau_{1} \geqslant t \right\}) \cdot P(\left\{ \tau_{2} \geqslant t \right\}) \\
			&= e^{-(\lambda_{1} + \lambda_{2})t}
		\end{align*}
		Hence, we can say that r.v. min$(\tau_{1}, \tau_{2})$ is exponentially
		distributed with mean $\frac{1}{\lambda_{1} + \lambda_{2}}$.

		For showing $P(\left\{ \lambda_{1} < \lambda_{2} \right\}) = \frac{\lambda_{1}}{\lambda_{1} + \lambda_{2}}$,
		we first know the event $\left\{ \lambda_{1} < \lambda_{2} \right\}$ is
		equivalent as $\left\{ \lambda_{1} < t \mid t = \lambda_{2} \right\}$. Hence,
		we can write,
		\begin{align*}
			P(\left\{ \lambda_{1} < \lambda_{2} \right\}) &= P(\left\{ \lambda_{1} < t \mid t = \lambda_{2} \right\}) \\
			&= \int_{0}^{\infty} P(\left\{ \lambda_{1} < t \right\}) f_{\tau_{2}}(t)dt \\
			&= \int_{0}^{\infty} (1 - e^{-\lambda_{1}t}) \lambda_{2}e^{-\lambda_{2}t}dt \\
			&= \lambda_{2} \int_{0}^{\infty} (1 - e^{-\lambda_{1}t}) e^{-\lambda_{2}t}dt \\
			&= \lambda_{2} \int_{0}^{\infty} (e^{-\lambda_{2}t} - e^{-(\lambda_{1} + \lambda_{2})})dt \\
			&= \lambda_{2} (\frac{e^{-\lambda_{2}t}}{\lambda_{2}} - \frac{e^{-(\lambda_{1} + \lambda_{2})}}{-(\lambda_{1} + \lambda_{2})}) \bigg|_{0}^{\infty} \\
			&= 1 - \frac{\lambda_{2}}{\lambda_{1} + \lambda_{2}} \\
			&= \frac{\lambda_{1}}{\lambda_{1} + \lambda_{2}}
		\end{align*} 
	\section*{Problem 5}
		\subsection*{Schwartz 2-6}
			\begin{enumerate}
				\item We need to obatin $(\lambda + \mu) p_{n} = \lambda p_{n-1} + \mu p_{n+1}$
					for $n \geqslant 1$. From Schwartz 2-12, we already have,
					\begin{gather*}
						p_{n}(t + \Delta t) = p_{n}(t) [ (1 - \lambda \Delta t)(1 - \mu \Delta t) + \mu \Delta t \cdot \lambda \Delta t + o(\Delta t) ] \\
						+ p_{n-1}(t) [ \lambda \Delta t (1 - \mu \Delta t) + o(\Delta t) ] \\
						+ p_{n+1}(t) [ \mu \Delta t (1 - \lambda \Delta t) + o(\Delta t) ]
					\end{gather*}
					We can factor the $(\Delta t)^{2}$ terms into $o(\Delta t)$ since they
					are usually very small,
					\begin{gather*}
						\implies p_{n}(t + \Delta t) = p_{n}(t) [1 - (\lambda + \mu) \Delta t] + p_{n-1}(t) (\lambda \Delta t) + p_{n+1}(t) (\mu \Delta t)
					\end{gather*}
					From Taylor Series,
					\begin{align*}
						p_{n}(t + \Delta t) &= p_{n}(t) + \frac{dp_{n}(t)}{dt} \Delta t \\
						\implies \frac{dp_{n}(t)}{dt} &= (-\lambda + \mu) p_{n}(t) + \lambda p_{n-1}(t) + \mu p_{n+1}(t)
					\end{align*}
					We want to find the steady state value of $p_{n}$, hence by setting $\frac{dp_{n}(t)}{dt} = 0$,
					\begin{align*}
						(\lambda + \mu) p_{n} = \lambda p_{n-1} + \mu p_{n+1}, n \geqslant 1
					\end{align*}
				\item From Fig. 2-11 in Schwartz textbook, we can use the Conservation
					of Rates (i.e., the rate going \textbf{in} a node equals to the rate
					going \textbf{out} of a node) to set up our equation which is just,
					\begin{align*}
						p_{n-1}\lambda + p_{n+1}\mu = p_{n}(\lambda + \mu)
					\end{align*}
			\end{enumerate}
		\subsection*{Schwartz 2-8}
			For the M/M/1 queue analysis, we have the stationary state probability $p_{n} = \rho^{n} p_{0}$,
			where $\rho = \frac{\lambda}{\mu}$.
			\begin{enumerate}
				\item Show that the stationary state probability satisfies Eq. 2-15.
				%TODO
				\item For showing $\lambda p_{n} = \mu p_{n+1}$ satifies Eq. 2-15, we
					can first start by examining Eq. 2-15,
					\begin{align*}
						(\lambda + \mu) p_{n} &= \lambda p_{n-1} + \mu p_{n+1} \\
						\implies \lambda p_{n} - \mu p_{n+1} &= \lambda p_{n-1} - \mu p_{n}
					\end{align*}
					As proved in class, $\lambda p_{n-1} - \mu p_{n} = 0$, hence,
					\begin{gather*}
						\lambda p_{n} - \mu p_{n+1} = 0 \\
						\implies \lambda p_{n} = \mu p_{n+1} 
					\end{gather*}
					As a result, the stationary state probability satisfies Eq. 2-15.
				\item We need to calculate $p_{0}$ for the M/M/1/N queu and show that $p_{n}$
					is given by Eq. 2-20 in Schwartz.
					From the class notes, we already showed that the stationary state
					probability for M/M/1/N queue is $p_{n} = (\frac{\lambda}{\mu})^n p_{0}$,
					therefore, by normalizing $p_{n}$, we get,
					\begin{align*}
						\sum_{n=0}^{N} p_{n} &= \sum_{n=0}^{N} (\frac{\lambda}{\mu})^n p_{0} \\
						&= p_{0} \sum_{n=0}^{N} \rho^{n} \\
						&= p_{0} \frac{1 - \rho^{N+1}}{1 - \rho} \\
						&= 1
					\end{align*}
					Hence,
					\begin{gather*}
						p_{0} = (\frac{1 - \rho^{N+1}}{1 - \rho})^{-1}
					\end{gather*}
					We can then substitute $p_{0}$ into the stationary state probability
					to obatin Eq. 2-20.
			\end{enumerate}

\section*{Simulation-Based Experiments}
	\subsection*{Problem 1}
	%TODO

	\subsection*{Problem 2}
	%TODO

	\subsection*{Problem 3}
	%TODO

\end{document}
